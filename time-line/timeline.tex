\documentclass{article}
\usepackage[left=3cm,right=3cm,top=2cm,bottom=2cm]{geometry} % page
                                                             % settings
\usepackage{amsmath} % provides many mathematical environments & tools
\usepackage[spanish]{babel}
\usepackage[doument]{ragged2e}

% Images
\usepackage{graphicx}
\usepackage{float}

% Code
\usepackage{listings}
\usepackage{xcolor}
\definecolor{gray}{rgb}{0.5,0.5,0.5}
\newcommand{\n}[1]{{\color{gray}#1}}
\lstset{numbers=left,numberstyle=\small\color{gray}}

\selectlanguage{spanish}
\usepackage[utf8]{inputenc}
\setlength{\parindent}{0mm}

\usepackage[bookmarks=true,
            bookmarksnumbered=false, % true means bookmarks in 
                                     % left window are numbered                         
            bookmarksopen=false,     % true means only level 1
            % are displayed.
            urlcolor=blue,
            colorlinks=true,
            linkcolor=webred]{hyperref}

\title{\textbf{Personajes y empresas relevantes en la historia \\ de las Bases de Datos}}
\author{David Cabezas \\ Emilio Hoyo \\ Patricia Córdoba \\ Iñaki Melguizo}
\date{\today}

\begin{document}
\maketitle

Timeline de los personajes y empresas relevantes en la historia de las
Bases de Datos, ordenados por la fecha de su primera aportación al
mundo de las bases de datos.

\begin{itemize}
\item \textbf{Herman Hollerith (1860 - 1929):} Desarrolló un tabulador
  electromagnético de tarjetas perforadas en \textbf{1884} para almacenaje y
  consulta de información.

\item \textbf{IBM y American Airlines:} Se aliaron para desarrollar
  SABRE en \textbf{1960}, un sistema operativo que manejaba las
  reservas de los vuelos, las transacciones e información de los
  pasajeros. IBM es una de las principales empresas en tecnología de
  bases de datos actualmente, desarrolló el modelo de bases de datos
  jerárquico (el más antiguo), además en \textbf{1974} desarrolló
  System R, la primera implementación de SQL.

\item \textbf{CODASYL, Conference on Data Systems Languages (1959):}
  Era un consorcio de industrias informáticas al que perteneció
  Charles Bachman. Formó el DBTG (Data Base Task Group).
  
\item \textbf{Charles Bachman (1924 - 2017):} Desarrolló IDS (Integrated
  Data Store) en \textbf{1964}, uno de los primeros sistemas de gestión de
  bases de datos.

\item \textbf{DBTG, Data Base Task Group (1965):} Fue un grupo de
  trabajo presidido por William Olle. En 1969 publicó las primeras
  especificaciones para el modelo de bases de datos en red.
  
\item \textbf{Edgar Frank Codd (1923 - 2003):} Inventó el modelo
  relacional para manejo de bases de datos en \textbf{1970}.
  
\item \textbf{Donald D. Chamberlin (1944) y Raymond F. Boyce (1947 -
    1974):} Desarrollaron SQL en \textbf{1974}, un lenguaje específico
  del dominio diseñado para administraar sistemas de gestión de bases
  de datos relacionales.
  
\item \textbf{Larry Ellison (1944), Bob Miner(1941 - 1994) y Ed Oates
    (1946):} Fundaron Oracle Corporation en \textbf{1977}, una
  compañía especializada en el desarrollo y marketing de software y
  tecnología de bases de datos. En particular, su propio sistema de
  gestión de bases de datos (Oracle Database).

\item \textbf{Microsoft Corporation (1975):} Fundada por Bill Gates y
  Paul Allen, desarrolló y comercializó en \textbf{1985} la primera
  versión de Excel, una aplicación de hojas de cálculo. Y en
  \textbf{1992} Access, un sistema de gestión de bases de datos
  relacionales. En \textbf{2007} lanzó LINQ, un componente de la
  plataforma Microsoft .NET que agrega capacidades de consulta a datos
  a los lenguajes .NET.

\item \textbf{MySQL AB (1995):} Fundada por Michael Widenius y David
  Axmark. Desarrolló MySQL en \textbf{1995}, un sistema de gestión de
  bases de datos relacionales de licencia libre. Fue comprada por
  Oracle Corporation en \textbf{2009}.

\item \textbf{World Wide Web Consortium (1994):} Fundada y dirigida
  por Tim Berners-Lee. En \textbf{2007} diseñó XQuery, un lenguaje de programación
  funcional para manipular colecciones de datos XML.

\end{itemize}

\subsection*{Referencias}
\begin{itemize}
\item \href{https://www.quickbase.com/articles/timeline-of-database-history}{https://www.quickbase.com/articles/timeline-of-database-history}
\item \href{https://www.wikipedia.org}{Wikipedia}
\item \href{http://histinf.blogs.upv.es/2011/01/04/historia-de-las-bases-de-datos}{http://histinf.blogs.upv.es/2011/01/04/historia-de-las-bases-de-datos}
\item \href{https://www.preceden.com/timelines/48236-historia-de--las-bases-de-datos}{https://www.preceden.com/timelines/48236-historia-de--las-bases-de-datos}
\end{itemize}

\end{document}
